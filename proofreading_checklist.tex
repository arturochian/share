\documentclass[11pt]{article}
\usepackage{a4wide}
\begin{document}

\title{R Journal Proof-reading Check List}
\author{by the R Journal Editorial Board}
\date{9 January 2012}
\maketitle

Your article has been accepted for publication in The R Journal. To
assist the editorial board with the production of your article, please
ensure that it conforms to the following check-list and send a revised
version to the handling editor.

\subsection*{Title, authors, abstract, sections}
\begin{itemize}
\item The article title should be title case (i.e.\ all words capitalized
   except for articles, conjunctions and prepositions).
\item The author list should start with ``by''.
\item The abstract should be created using the \verb+\abstract{}+
   environment provided by the R Journal style file \texttt{RJournal.sty}.
\item The abstract should not contain any references.
\item  Section and subsection titles should be in sentence case (i.e.\ only
   the first word and proper names capitalized)
\item Use the \verb+\section+ and \verb+\subsection+ commands for sections and 
      subsections, not the starred versions (\verb+\section*+
      and \verb+\section*+). The result may look the same but your
      sections will not be properly indexed in the PDF file if you use
      the starred versions.
\item All authors must have an affiliation at the end of the article,
      with the email address of at least the first author.
\end{itemize}

\subsection*{Software markup and citation}
\begin{itemize}
\item R itself should not have a citation or bibliography entry.
\item All other software (including R packages) must have a citation after
  the first mention in the body text (not the abstract) and a
  corresponding entry in the bibliography.
\item The first mention of a CRAN package in the body text (not the title or
  abstract) should be marked up with \verb+\CRANpkg+. This
  provides a hyper-link to the package homepage on CRAN.
\item Other than the above, names of R packages should be marked up with 
  \verb+\pkg+, including in the title (This ensures that the
  package name is correctly formatted in the table of contents).
\item Programming languages (including R), and other sofware
  (e.g. Stata, Matlab) do not have special markup but appear in the
  same font as the rest of the body text.
\end{itemize}

\section*{R code}

\begin{itemize}
\item Use \verb+\code{}+ to mark up functions, objects, argument names etc.\
appearing in the text.  Class names should appear in double quotes.
\item Ensure that code listings do not overflow the right hand margin.
\item Use indentation and spacing to improve the readability of code
  listings where possible, e.g. put spaces after commas and around
  equals signs.
\end{itemize}

\section*{General LaTeX issues}

\begin{itemize}
\item Put a trailing backslash -- \verb+.\+ -- after periods that are not at 
  the end of a sentence (when using e.g., i.e., etc.).
\item Use \verb+\url+ for all URLs. The file \texttt{RJwrapper.tex} 
  loads the LaTeX package ``url'' for you.
\item Use booktabs format for tables (e.g.\ \verb+\toprule+, 
  \verb+\midrule+, \verb+bottomrule+ rather than \verb+\hline+).
  The file \texttt{RJwrapper.tex} loads the LaTeX package ``booktabs''
  for you.
\item Avoid superfluous LaTeX formatting, e.g.
  \begin{itemize}
  \item Bold in figure captions,
  \item Use of \verb+\setlength{itemsep}+ to change vertical spacing
  \end{itemize}
\end{itemize}

\section*{Mathematics}

\begin{itemize}
\item Punctuation after equations must be consistent (i.e.\ you may
  use it in all equations, or not at all).
\item Formatting tools provided by the American Mathematical Society
  are preferred to the default LaTeX formatting. The
  \texttt{RJwrapper.tex} file loads the LaTeX packages ``amsmath'' and
  ``amssymb'' for you. In particular,
  \begin{itemize}
  \item Use \verb+pmatrix+ for matrices, not \verb+array+
  \item Use \verb+align+ or \verb+alignat+ for stacked equations,
    not \verb+eqnarray+.
  \end{itemize}
\item In displayed formulae, use \verb+\left(+ and \verb+\right+, not
  \verb+(+ and \verb+)+ to ensure correct sizing of braces.
\item Do not leave empty lines before displayed formulae, and only
  leave a blank line after a displayed formula if there is a paragraph
  break.
\item Use \verb+$-5$+ not \verb+-5+ for a true minus prefix.
\end{itemize}

\section*{Bibliography}

\begin{itemize}
\item The bibliography should be created with BibTeX, and the BibTeX
  sources included in the submission.
\item Bibliography entries should not contain any markup specific to
  The R Journal. For example, package names should not be marked up
  with \verb+\pkg+ in the bibliography.
\item Acronyms and proper names in titles should be protected with
  braces \verb+{}+ to ensure that they appear properly capitalized
  in the bibliography. Likewise, corporate authors should be
  protected to ensure that they appear verbatim and are not abbreviated.
\item Do not override the default bibliography style.
\end{itemize}

\end{document}
