\documentclass[a4paper]{report}
\usepackage{RJournal}
\usepackage[round]{natbib}
\usepackage{booktabs}
\bibliographystyle{abbrvnat}

\begin{document}

%\renewcommand{\sectionmark}[1]{\markright{#1}{}}
%\fancyhf{}
%\fancyhead[LO,RE]{\textsc{\rightmark}}
%\fancyhead[RO,LE]{\textsc{Instructions for Authors}}
\fancyhf{}
\fancyhead[LO,RE]{\textsc{Instructions for Authors}}
\fancyhead[RO,LE]{\thepage}

\begin{article}
\title{Instructions for Authors}

\author{by The R Journal Editors}

\maketitle

\abstract{
\emph{The R Journal} is compiled using \LaTeX{} and authors are required
to submit their articles as \LaTeX{} documents. Here we provide authors
with information for preparing submissions to the \emph{Journal}.
}

\section{Introduction}

\emph{The R Journal} is the refereed journal of the R Project for Statistical
Computing \citep{R:Ihaka+Gentleman:1996}.
It features short to medium length articles covering topics
that might be of interest to users or developers of R, including:

\begin{description}

\item[Add-on packages:] Short introductions to R extension packages.

\item[Programmer's Niche:] Hints for programming in R.

\item[Help Desk:] Hints for newcomers explaining aspects of R that might
  not be obvious from reading the manuals and FAQs.

\item[Applications:] Demonstrating how a new or existing technique can be
applied in an area of current interest using R, providing a fresh view
of such analyses in R that is of benefit beyond the specific application.

\end{description}

\emph{The R Journal} intends to reach a wide audience and have a fast-track
but thorough review process. Papers are expected to be reasonably short,
clearly written, not too technical, and of course focused on R.

Authors of refereed articles should take care to
\begin{itemize}
\item put their contribution in context, in particular discuss related R
  functions or packages;
\item explain the motivation for their contribution;
\item provide code examples that are reproducible.
\end{itemize}

\emph{The R Journal} also has a \emph{News and Notes} section,
including information on:

\begin{description}

\item[Changes in R:] New features of the latest release.

\item[Changes on CRAN:] New add-on packages, manuals, binary distributions,
mirrors, etc.

\item[News from the Bioconductor project:] Latest developments from 
\url{www.bioconductor.org}.

\item[R Foundation News:] Donations to and new members of The R Foundation.

\item[Conferences:] Upcoming R-related conferences and reports from
conferences.

\end{description}

The purpose of this document is to describe to all prospective authors how
to prepare a submission for \emph{The R Journal}.

\section{Preparing a submission}

Please send submissions to regular columns (Programmer's Niche, Help Desk)
to the respective column editor, all other submissions to the Editor-in-Chief
or a member of the editorial board.

The following files provide a template for preparing an article for submission
to \emph{The R Journal}:

\begin{description}

\item[\LaTeX{} style file:] \file{RJournal.sty}.

\item[Master \LaTeX{} file:] \file{RJwrapper.tex}. This includes the file
  \file{RJtemplate.tex}, which is not itself a complete \LaTeX{} document
  (it has no \verb|\begin{document}| or \verb|\end{document}|).

\item[Article template:] \file{RJtemplate.tex}.

\end{description}

Running \code{pdflatex} on \file{RJwrapper.tex} a couple of times
(to get the references right) will produce
\file{RJwrapper.pdf}, which shows how the template file would be
typeset in an \emph{R Journal} issue.

\file{RJtemplate.tex} should be modified to contain the body of your
article and renamed according to the author's or authors' surnames. For
example, an article by John Chambers would be in the file \file{Chambers.tex},
and one by Bill Venables and Brian Ripley would be in the file
\file{VenablesRipley.tex}. 

\file{RJwrapper.tex} must then be modified to include your article; all \LaTeX{}
packages required by your article should be loaded in this file. Both
\file{.tex} files should be submitted, along with the compiled
\file{RJwrapper.pdf} and all necessary figure files (the \strong{References} section
describes how a bibliography may be included within the article \file{.tex} file).

Do not include style files for other latex packages needed by your article.

\section{Language}

Articles in \emph{The R Journal} are written in English. We accept British and American
spelling along with other national variations. We encourage authors for whom English is
not their first language to have their papers edited by a competent copy-editor. We
encourage all authors to conform to accepted norms of grammar and style, and to avoid
sexist language, such as the use of `he' for individuals of indefinite gender.

\section{Marking text}

The \LaTeX{} style file \file{RJournal.sty} provides a much simplified version of the
commands for marking words and phrases used by
Texinfo\footnote{\url{http://www.gnu.org/software/texinfo/}} (but note that the \LaTeX{} special
characters still need special treatment). Please use these commands and the other
mark-up facilities described in this section rather than attempting to format
output and other elements visually. Unless it is absolutely necessary, please refrain from
introducing additional idiosyncratic mark-up---for example, for programming languages.

The commands provided are:

\begin{description}

\item[\code{$\backslash$ \\code\{\var{sample-code}\}}]
indicates text that is a literal example of a piece of a program.
For example, \verb|\code{rows <- nrow(X)}| is typeset as
\code{rows <- nrow(X)}. The \verb|\code| command should also
be used for keyboard input and the names of objects, functions and
arguments. Class names should be quoted; for example \verb|\code{"lm"}| is
typeset as \code{"lm"}. 

\item[\code{$\backslash$ \\samp\{\var{text}\}}]
indicates text that is a literal example of a sequence of
characters. It should be used whenever parts of inline code
could be confused with text, for example \verb|\samp{R CMD check}| is
typeset as \samp{R CMD check} and  e.g. \verb|\samp{...}| would give
\samp{...}. 

\item[\code{$\backslash$ \\file\{\var{file-name}\}}]
indicates the name of a file. For example,
\verb|\file{RJwrapper.tex}| is typeset as \file{RJwrapper.tex}.

\item[\code{$\backslash$ \\dfn\{\var{term}\}}]
indicates the introductory or defining use of a term.
For example, \verb|\dfn{environment}| is typeset as
\dfn{environment}.

\end{description}

We also provide the following markup:

\begin{description}
\item[\code{$\backslash$ \\strong}]
emphasizes text more strongly than  \verb|\emph|.
For example, \verb|\strong{Note:}| is typeset as \strong{Note:}.

\item[\code{$\backslash$ \\pkg}]
indicates an R package. For example,
\verb|\pkg{MASS}| is typeset as \pkg{MASS}.

\item[\code{$\backslash$ \\CRANpkg}]
indicates an R package on CRAN, and includes a hyper-link to the
corresponding web page. For example,
\verb|\CRANpkg{Rcpp}| is typeset as \CRANpkg{Rcpp}.

\item[\code{$\backslash$ \\url}] indicates a URL.
For example,\\ \verb|\url{http://cran.r-project.org/}| is typeset
as \url{http://cran.r-project.org/}.

\end{description}

Note that no markup is necessary to typeset R. Likewise, no markup
should be used to typeset the names of external software. In particular,
the \verb|\pkg| command is reserved for R packages.

\subsection{Quotations and examples}

In addition to the standard \LaTeX{} environments for quotations and examples (such
as \verb|quote|, \verb|quotation|, \verb|flushleft|, \verb|center| and \verb|flushright|), the
\pkg{\filename} package provides the following environments:

\begin{description}

\item[\code{example}]
is used to illustrate code, commands, and the like.  The text is printed in a
fixed-width font, and indented but not filled.

\item[\code{smallexample}]
is similar to \code{example}, except that text is typeset in a smaller
font.

\end{description}

These are patterned after the Texinfo environments with the same
names.  In particular, \verb|\{|, \verb|\}|, and \verb|\\| retain their ``usual'' meanings
and are not treated verbatim, which is not optimal for displaying R
code or output.  Hence, we also provide a \verb|smallverbatim| environment
which works like \verb|verbatim| but uses a smaller font for typesetting.

\subsection{Sectioning, titles, and abstract}

Use only \verb|\section| and \verb|\subsection| commands, not
\verb|\section*| or \verb|\subsection*|.

The title of the article should be set with initial capitals, as in
\verb|\title{Drawing Diagrams with R}|. Only the initial word of
section and subsection titles should be capitalized; for example,
\verb|\section{Starting at the end}|. 

If the title includes a package name, the name should be formatted
with the \verb|\pkg| command. This ensures that the package name is
correctly typeset when it appears in the Table of Contents of \emph{The R
Journal}. Note that \verb|\pkg| is the only markup that should be used
inside a title.

Every article should include an abstract of no more than 150 words. The abstract
is entered with the \verb|\abstract| command, and should appear immediately
after \verb|\maketitle| at the beginning of the article. The abstract
should not contain any citations or references.

\subsection{Author information}

Authors' names only should be given at the beginning of the article,
following the title, using the \verb|\author| command. The list of
authors should begin with the word ``by''. All other information is
given in the `signature block' at the end of the article (see
immediately below). For example,
\verb|\author{by Ross Ihaka and Robert Gentleman}|.

The article should end with a signature block giving contact information
for each author. For example

\begin{verbatim}
\address{Paul Murrell\\
  Department of Statistics\\
  The University of Auckland\\
  New Zealand}\\
\email{paul@stat.auckland.ac.nz}
\end{verbatim}

\subsection{Mathematics}

\emph{The R Journal} does not prescribe specific \LaTeX{} mark-up for mathematics: Use
mark-up that is conventional in your field. We do, however, encourage authors to
follow sound \LaTeX{} practices. 

\begin{itemize}

\item For example, use proper mathematical operators: 
Do not write \verb|log(x)|, which will be typeset as $log(x)$, but rather \verb|\log(x)|,
which will appear as $\log(x)$. 

\item Similarly, use \verb|\left| and \verb|\right| with
delimiters in mathematical expressions in preference to bare delimiters:
Do not write

\begin{verbatim}
\sum_{1=1}^{n}(X_{i}^{\prime} - 
  \overline{X}^{\prime})^2
\end{verbatim}

which will be typeset as $\sum_{1=1}^{n}(X_{i}^{\prime} - \overline{X}^{\prime})^2$,
but rather

\begin{verbatim}
\sum_{1=1}^{n}\left(X_{i}^{\prime} - 
  \overline{X}^{\prime}\right)^2
\end{verbatim}

which will appear as $\sum_{1=1}^{n}\left(X_{i}^{\prime} - \overline{X}^{\prime}\right)^2$.

\end{itemize}

%The commands \verb|\P|, \verb|\E|, \verb|\VAR|, \verb|\COV|, and \verb|\COR| produce symbols
%for probability, expectation, variance, covariance, and correlation.
%For example, Chebyshev's inequality
%\begin{displaymath}
%\P\left(\left|\xi - \E\xi\right| > \lambda\right) \le \frac{\VAR\left(\xi\right)}{\lambda^2}
%\end{displaymath}
%can be coded as
%\begin{verbatim}
%\P\left(\left|\xi - \E\xi\right| > \lambda\right)
%  \le \frac{\VAR\left(\xi\right)}{\lambda^2}
%\end{verbatim}
%
%The symbols
%\begin{displaymath}
%\mathbb{N}\quad\mathbb{Z}\quad\mathbb{Q}\quad\mathbb{R}\quad\mathbb{C}
%\end{displaymath}
%for the positive integers, the integers, and the rational, real and
%complex numbers, respectively, can be obtained using \verb|\mathbb| from
%package \pkg{amsfonts} as
%\verb|\mathbb{N}|, \verb|\mathbb{Z}|, \verb|\mathbb{Q}|, \verb|\mathbb{R}|,
%and \verb|\mathbb{C}|.

\section{Figures and tables}

Currently, \emph{The R Journal} is typeset in two columns.  By
default, figures and tables will occupy only one column (see Figure
\ref{figure:onecolfig}). Single-column figures and tables are not
floating, so will appear exactly where they are placed in the
text. This can leave large amounts of unwanted white space, which may
require some manual adjustment to the placement of the table or figure
code.

The \verb|figure*| or \verb|table*| environments create a figure or
table that spans both columns (see Figure \ref{figure:bibexample}).
Two-column figures and tables have limited floating capability: they
may appear at the top of the next page or on a page of their own.

\begin{figure}
\vspace*{.1in}
\framebox[\textwidth]{\hfill \raisebox{-.45in}{\rule{0in}{1in}}
                      A picture goes here \hfill}
\caption{\label{figure:onecolfig}
A normal figure only occupies one column.}
\end{figure}

Horizontal lines in tables should use commands from the
booktabs package, i.e.\ \verb|\toprule| for the top of the table,
\verb|\bottomrule| for the bottom of the table, and \verb|\midrule|
for any horizontal lines within the table (see Table \ref{table:onecoltab}).

\begin{table}
\centering
\begin{tabular}{lcc}
\toprule
   & Left & Right \\
\midrule
Up & 1  &  2 \\
Down &  3 & 4 \\
\bottomrule
\end{tabular}
\caption{\label{table:onecoltab}
A simple table with booktabs formatting.}
\end{table}

\section{References}
\label{sec:references}

The standard way to produce citations for \emph{The R Journal} is via the
\verb|\citep| and \verb|\citet| commands (and their relatives)
and a \file{.bib} file that contains the
references in {\sc Bib}\TeX{} format.\footnote{We use the \pkg{natbib}
package for citations.}  The citation in the first
paragraph of this style guide is of the form
\verb|\citep{R:Ihaka+Gentleman:1996}|.  Figure \ref{figure:bibexample}
shows an example file called \file{example.bib} which contains
a single reference.

A bibliography is produced from \file{example.bib}
by placing the following line in \file{RJtemplate.tex} (or
whatever you end up calling it):
\begin{verbatim}
\bibliography{example}
\end{verbatim}
and running \code{pdflatex} then \code{bibtex} on the file
\file{RJwrapper.tex}, then running \code{pdflatex} as many times as
necessary until \LaTeX{} stops complaining about undefined citations.

\begin{figure*}[b]
\begin{center}
\begin{boxedverbatim}
@ARTICLE{R:Ihaka+Gentleman:1996,
  AUTHOR = {Ross Ihaka and Robert Gentleman},
  TITLE = {R: A Language for Data Analysis and Graphics},
  JOURNAL = {Journal of Computational and Graphical Statistics},
  YEAR = 1996,
  VOLUME = 5,
  NUMBER = 3,
  PAGES = {299--314},
  URL = {http://www.amstat.org/publications/jcgs/}
}
\end{boxedverbatim}
\end{center}
\caption{\label{figure:bibexample}
The contents of a file called \file{example.bib}.  This figure
uses the \code{figure*} environment to span
two columns.}
\end{figure*}

BibTeX will ignore capitalization in titles, unless words are
protected inside curly braces, e.g.\ \verb|R| will appear as ``r'',
whereas \verb|{R}| will appear correctly as ``R''. Ensure that proper
names in titles are protected. Similarly, in the author field,
corporate authors will not appear correctly unless protected, e.g. a
bibtex entry with
\begin{verbatim}
AUTHOR = {The R Journal Editors}
\end{verbatim}
will appear in the bibliography as ``TRJ Editors'' and should be
protected with double braces:
\begin{verbatim}
AUTHOR = {{The R Journal Editors}}
\end{verbatim}

\subsection{Citing packages}

The first time a package is cited in the text, excluding the abstract,
it should be followed by a formal citation, such as the one generated
by the \verb|citation()| function in R. The first citation of a CRAN
package should use \verb|\CRANpkg|. Further citations use \verb|\pkg|
and need not be followed by a citation.

The BibTeX entry for an R package should not use the \verb|\pkg|
command to format the package name in the \code{TITLE} field.

\subsection{Citing R}

Articles in \emph{The R Journal} do not include a citation of R itself.

\section{Summary}

The steps involved in preparing an article for submission to \emph{The
  R Journal} are as follows:

\begin{itemize}

\item Download \file{RJwrapper.tex}, \file{RJtemplate.tex},
and \file{RJournal.sty}.

\item Rename \file{RJtemplate.tex} using the author's or authors' names
(say, \file{Yourname.tex}), and replace its
contents with the contents of your article.

\item Create a \file{Yourname.bib} BibTeX file and add
\verb|\bibliography{Yourname}| at the end of \file{Yourname.tex}.

\item Modify \file{RJwrapper.tex} to include \file{Yourname.tex} rather
than \file{RJtemplate.tex}. Include all necessary \LaTeX{} \verb|\usepackage|
commands in the modified \file{RJwrapper.tex}.

\item Run \code{pdflatex} on \file{RJwrapper.tex} a couple of times
(until all figure references are resolved) to produce \file{RJwrapper.pdf}.

\item Iterate until \file{RJwrapper.pdf} looks right.

\item Then submit
  \begin{itemize}
  \item The modified \file{RJwrapper.tex};
  \item \file{RJwrapper.pdf};
  \item \file{Yourname.tex};
  \item \file{Yourname.bib};
  \item and all necessary figure files.
  \end{itemize}
\end{itemize}

\section{Acknowledgment}

Parts of this style guide were adapted from documentation originally
prepared by Kurt Hornik and Friedrich Leisch for the \emph{R Journal}
\LaTeX{} style file.

%\renewcommand{\bibsection}{\section{Bibliography}}

\begin{thebibliography}{1}
\expandafter\ifx\csname natexlab\endcsname\relax\def\natexlab#1{#1}\fi
\expandafter\ifx\csname url\endcsname\relax
  \def\url#1{{\tt #1}}\fi
\bibitem[Ihaka and Gentleman(1996)]{R:Ihaka+Gentleman:1996}
R.~Ihaka and R.~Gentleman.
\newblock R: A language for data analysis and graphics.
\newblock {\em Journal of Computational and Graphical Statistics}, 5\penalty0
  (3):\penalty0 299--314, 1996.
\newblock URL \url{http://www.amstat.org/publications/jcgs/}.

\end{thebibliography}
\end{article}

\end{document}


